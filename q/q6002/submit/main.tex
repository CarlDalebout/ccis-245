\input{thispreamble.tex}

\renewcommand\AUTHOR{cadalebout1@cougars.ccis.edu} % CHANGE TO YOURS

\begin{document}
\topmattertwo

%------------------------------------------------------------------------------
\nextq
Write a class \verb!Time! that has private members
\verb!hh_!,
\verb!mm_!,
\verb!ss_!,
for hours, minutes, seconds in 24-hour format, i.e.,
hours is in the range 0..23.
Write enough public methods so that you can 
create an array \verb!t! of 10 \verb!Time! objects
so that
\verb!t[0]! models the time 00:00:00,
\verb!t[1]! models the time 00:01:00,
\verb!t[2]! models the time 00:02:00,
...
\verb!t[9]! models the time 00:08:00.
Print all the object in \verb!t!.
\\
\textsc{Answer:}\vspace{-2mm}
\begin{answercode}
#include <iostream>
#include <iomanip>

class Time
{
public:
    Time(int h = 0, int m = 0, int s = 0)
    :hh_(h), mm_(m), ss_(s)
    {}

    void setHHMMSS(int hh, int mm, int ss)
    {
        hh_ = hh;
        mm_ = mm;
        ss_ = ss;
    }

    void println()
    {
        std::cout << std::setw(2) << std::setfill('0') << hh_ << ':'
                  << std::setw(2) << std::setfill('0') << mm_ << ';'
                  << std::setw(2) << std::setfill('0') << ss_ << std::endl;
    }

private:
int hh_, mm_, ss_;
};

int main()
{
    Time t[10];// Declare t to be an array of 10 Time objects

    for(int i = 0; i < 10; ++i)
    {
        t[i].setHHMMSS(0,i,0);
        t[i].println();
    }

    return 0;
}
\end{answercode}

%------------------------------------------------------------------------------
\nextq
This is the same as the previous question \textit{except}
that \verb!t! is an array of \verb!Time! pointers.
You can copy the \verb!Time! class from above to the answer here.
\\
\textsc{Answer:}\vspace{-2mm}
\begin{answercode}
#include <iostream>
#include <iomanip>

class Time
{
public:
    Time(int h = 0, int m = 0, int s = 0)
    :hh_(h), mm_(m), ss_(s)
    {}

    void setHHMMSS(int hh, int mm, int ss)
    {
        hh_ = hh;
        mm_ = mm;
        ss_ = ss;
    }

    void println()
    {
        std::cout << std::setw(2) << std::setfill('0') << hh_ << ':'
                  << std::setw(2) << std::setfill('0') << mm_ << ';'
                  << std::setw(2) << std::setfill('0') << ss_ << std::endl;
    }

private:
int hh_, mm_, ss_;
};

int main()
{
    Time *t = new Time[10];// Declare t to be an array of 10 Time objects

    for(int i = 0; i < 10; ++i)
    {
        t[i].setHHMMSS(0,i,0);
        t[i].println();
    }
    delete[] t;
    return 0;
}

\end{answercode}

%------------------------------------------------------------------------------
\nextq
Wumpus lives in the Wumpus world, which is a 4-by-4 grid.
For instance in the following
\begin{console}[frame=single,fontsize=\footnotesize]
+-+-+-+-+
| | | | |
+-+-+-+-+
| | | |W|
+-+-+-+-+
| | | | |
+-+-+-+-+
| | | | |
+-+-+-+-+
\end{console}
Wumpus is at row 1, column 3.
You are given this:
\begin{console}[frame=single,fontsize=\footnotesize]
// file: WumpusWorld.h
#ifndef WUMPUSWORLD_H
#define WUMPUSWORLD_H

#include <iostream>

class WumpusWorld
{
public:
    init();              // initialize so that Wumpus is at (0,0).
    println();           // print according to the above format
    move_wumpus();       // randomize a direction rand() % 4 where 0,1,2,3
                         // are N,S,E,W for Wumpus. Of course Wumpus
                         // must stay in the world.
                         // If wumpus tries to move N but its row is 0,
                         // wumpus stays put.
private:
    char world_[4][4];
};
#endif
\end{console}
Do not add anything to the above.
You want to make Wumpus wonder around like this:
\begin{console}[frame=single,fontsize=\footnotesize]
#include <iostream>
#include <ctime>
#include <cstdlib>
#include "WumpusWorld.h"

int main()
{
    srand((unsigned int) time(NULL));
    
    WumpusWorld ww;
    ww.init();
    ww.println();
    for (int i = 0; i < 5; ++i)
    {
        ww.move_wumpus();
        ww.println();
    }
    return 0;
}
\end{console}
Complete the following cpp file that contains the implementation
of the methods declared in the WumpusWorld class.
\\
\textsc{Answer:}\vspace{-2mm}
\begin{answercode}
// file: WumpusWorld.cpp

void WumpasWorld::init()
{
    for(int i = 0; i < 4; ++i)
    {
        for(int j = 0; j < 4; ++j)
        {
            world_[i][j] = ' ';
        }
    }
    world_[0][0] = 'W';
}

void WumpasWorld::move_wumpas()
{
        int direction = rand() % 4;

        switch (direction)
        {
            case 0://north
                for(int i = 1; i < 4; ++i)
                {
                    for(int j = 0; j < 4; j++)
                    {
                        if(world_[i][j] = 'W')
                        {
                            world_[i][j] = " ";
                            world_[i-1][j] = 'W';
                        }
                    }
                }
                break;
            case 1:
                for(int i = 0; i < 3; ++i)
                {
                    for(int j = 0; j < 4; j++)
                    {
                        if(world_[i][j] = 'W')
                        {
                            world_[i][j] = " ";
                            world_[i+1][j] = 'W';
                        }
                    }
                }
                break;
            case 2:
                for(int i = 0; i < 4; ++i)
                {
                    for(int j = 1; j < 4; j++)
                    {
                        if(world_[i][j] = 'W')
                        {
                            world_[i][j] = " ";
                            world_[i][j-1] = 'W';
                        }
                    }
                }
                break;
            case 3:
                for(int i = 0; i < 4; ++i)
                {
                    for(int j = 0; j < 3; j++)
                    {
                        if(world_[i][j] = 'W')
                        {
                            world_[i][j] = " ";
                            world_[i][j+1] = 'W';
                        }
                    }
                }
                break;
        }
}

void WumpasWorld::println()
{
    std::cout << "+-+-+-+-+\n";
    for(int i = 0; i < 4; ++i)
    {
        for(int j = 0; j < 4; ++j)
        {
            std::cout << '|' << world_[i][j];
        }
        std::cout << "|\n";
        std::cout << "+-+-+-+-+\n";
    }
}

\end{answercode}


\newpage
\input{instructions.tex}
\end{document}
