\input{thispreamble.tex}

\renewcommand\AUTHOR{cadalebout1@cougars.ccis.edu} % CHANGE TO YOURS

\begin{document}
\topmattertwo

The questions below uses this class: 
\begin{Verbatim}[frame=single,fontsize=\footnotesize]
class X
{
public:
    X(int id)
        : id_(id)
    {}
    ~X()
    {
        std::cout << id_ << " died\n";
    }
    int id_
};
\end{Verbatim}

%------------------------------------------------------------------------------
\nextq
What is the output of this code fragment?
The answer you write down must match the output exactly (watch out for
spaces and newlines).
\begin{Verbatim}[frame=single,fontsize=\footnotesize]
int main()
{
    X x(42);
    for (int i = 0; i < 3; ++i)
    {
        X a(i);
    }
    return 0;
}
\end{Verbatim}
\textsc{Answer:}\vspace{-2mm}
\begin{answercode}
0 died
1 died
2 died
42 died
\end{answercode}

%------------------------------------------------------------------------------
\nextq
What is the output of this code fragment?
The answer you write down must match the output exactly (watch out for
spaces and newlines).
\begin{Verbatim}[frame=single,fontsize=\footnotesize]
void f(X x)
{
    std::cout << "f\n";
    x.id_ = 0;
    return;
}
void g(const X & x)
{
    std::cout << "g\n";
    x.id_ = 1;
    return;
}
X h(const X & x)
{
    std::cout << "h\n";
    x.id_ = 2;
    return x;
}
int main()
{
    X x(42);
    f(x);
    g(x);
    X y = h(x);
    y.id_ = 3;
    return 0;
}
\end{Verbatim}
\textsc{Answer:}\vspace{-2mm}
\begin{answercode}
f
g
h
3 died
1 died
\end{answercode}


%------------------------------------------------------------------------------
\nextq
What is the output of this code fragment?
The answer you write down must match the output exactly (watch out for
spaces and newlines).
\begin{Verbatim}[frame=single,fontsize=\footnotesize]
int main()
{
    X * p = new X(0);
    X * q = new X(1);
    X * r = new X(2);
    delete p;
    delete r;
    return 0;
}
\end{Verbatim}
\textsc{Answer:}\vspace{-2mm}
\begin{answercode}
0 died
2 died
\end{answercode}

\newpage
\input{instructions.tex}
\end{document}
