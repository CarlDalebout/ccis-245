\input{thispreamble.tex}

\renewcommand\AUTHOR{cadalebout1@cougars.ccis.edu} % CHANGE TO YOURS

\begin{document}
\topmattertwo

%------------------------------------------------------------------------------
\nextq
The class \verb!Date! has member functions
\verb!add_y(int inc_y)!,
\verb!add_m(int inc_m)!, and 
\verb!add_d(int inc_d)! that respectively increments the
year, month, and day of a \verb!Date! object.
Complete the following function that
increments the year, month, day of a Date object.
\\
\textsc{Answer:}\vspace{-2mm}
\begin{answercode}
void Date::add_y_m_d(int inc_y, int inc_m, int inc_d)
{
        *this.add_y(inc_y);
        *this.add_m(inc_m);
        *this.add_d(inc_d);
}
\end{answercode}

%------------------------------------------------------------------------------
\nextq
You are given
\begin{console}[fontsize=\footnotesize,frame=single]
class Position
{
private:
    int x_, y_;
};
\end{console}
Add enough code to the above so that the following works:
\begin{console}[fontsize=\footnotesize,frame=single]
Position p;
p.set_x(0); // p.x_ is set to 0
p.set_y(1); // p.y_ is set to 1
p.print();  // print "<0, 1>" (without double quotes)
\end{console}
\\
\textsc{Answer:}\vspace{-2mm}
\begin{answercode}
class Position
{
public:

void set_x(int v){ x_ = v; }
void set_y(int v){ y_ = v; }

void print()
{
        std::cout << '<' << x_ << ", " << y_ << ">\n";
}

private:
    int x_, y_;
};
\end{answercode}

%------------------------------------------------------------------------------
\nextq
You are given
\begin{console}[fontsize=\footnotesize,frame=single]
class Array
{
public:
    push_back(int v)
    {
        if (size_ < capacity_)
        {
            x_[size_] = v;
            ++size_;
        }
    }
private:
    int x_[1024];
    int size_;
    int capacity_;
};
\end{console}
Add a method \verb!is_ascending()! that will return true
of \verb!x_[0]!, ..., \verb!x_[size_  - 1]! is in ascending order.
For instance
\begin{console}[fontsize=\footnotesize,frame=single]
Array a;
a.push_back(1);
a.push_back(3);
a.push_back(3);
a.push_back(5);
std::cout << a.is_ascending() << '\n'; // prints 1
\end{console}
\\
\textsc{Answer:}\vspace{-2mm}
\begin{answercode}
class Array
{
public:
    void push_back(int v)
    {
        if (size_ < capacity_)
        {
            x_[size_] = v;
            ++size_;
        }
    }

    bool is_ascending()
    {
        if (size_ == 1)
           return true;
        for(int i = 1; i < size_; ++i)
        {
                if(x_[i] < x[i-1])
                         return false;
        }
        return true;
    }

private:
    int x_[1024];
    int size_;
    int capacity_;
};
\end{answercode}


\newpage
\input{instructions.tex}
\end{document}
