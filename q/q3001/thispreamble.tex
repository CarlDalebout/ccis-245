\newcommand\COURSE{ciss245}
\newcommand\ASSESSMENT{q3001}
\newcommand\ASSESSMENTTYPE{Quiz}
\newcommand\POINTS{\textwhite{xxx/xxx}}

\makeatletter
\DeclareOldFontCommand{\rm}{\normalfont\rmfamily}{\mathrm}
\DeclareOldFontCommand{\sf}{\normalfont\sffamily}{\mathsf}
\DeclareOldFontCommand{\tt}{\normalfont\ttfamily}{\mathtt}
\DeclareOldFontCommand{\bf}{\normalfont\bfseries}{\mathbf}
\DeclareOldFontCommand{\it}{\normalfont\itshape}{\mathit}
\DeclareOldFontCommand{\sl}{\normalfont\slshape}{\@nomath\sl}
\DeclareOldFontCommand{\sc}{\normalfont\scshape}{\@nomath\sc}
\makeatother

\input{myquizpreamble}
\input{yliow}
\input{\COURSE}
\textwidth=6in

\renewcommand\TITLE{\ASSESSMENTTYPE \ \ASSESSMENT}

\newcommand\topmattertwo{
\topmatter
\score \\ \\
Open \texttt{main.tex} and enter answers (look for
\texttt{answercode}, \texttt{answerbox}, \texttt{answerlong}).
Turn the page for detailed instructions.
To rebuild and view pdf, in bash shell execute \texttt{make}.
To build a gzip-tar file, in bash shell execute \texttt{make s} and
you'll get \texttt{submit.tar.gz}.
}

\newcommand\tf{T or F or M}
\newcommand\answerbox[1]{\textbox{\phantom{|}\hspace{-4mm}#1}}
\newcommand\codebox[1]{\begin{console}#1\end{console}}

\usepackage{pifont}
\newcommand{\cmark}{\textred{\ding{51}}}
\newcommand{\xmark}{\textred{\ding{55}}}

\newcounter{qc}
\newcommand\nextq{
%\newpage
\addtocounter{qc}{1}
Q{\theqc}.
}

\DefineVerbatimEnvironment%
 {answercode}{Verbatim}
 {frame=single,fontsize=\footnotesize}

\newenvironment{largebox}[1]{%
 \boxparone{#1}
}
{}

\usepackage{environ}
\let\oldquote=\quote
\let\endoldquote=\endquote
\let\quote\relax
\let\endquote\relax

% ADDED 2021/09/09
\renewcommand\boxpar[1]{
 \[
  \framebox[\textwidth][c] {
   \parbox[]{\dimexpr\textwidth - 0.25cm} {#1}
  }
 \]
}

\NewEnviron{answerlong}%
  {\vspace{-1mm} \global\let\tmp\BODY\aftergroup\doboxpar}

\newcommand\doboxpar{%
  \let\quote=\oldquote
  \let\endquote=\endoldquote
  \boxpar{\tmp}
}

\newenvironment{mcq}[7]%
{% begin code
#1 \dotfill{#2}
 \begin{tightlist}
 \item[(A)] #3
 \item[(B)] #4
 \item[(C)] #5
 \item[(D)] #6
 \item[(E)] #7 
 \end{tightlist}
}%
{% end code
} 

\renewcommand\EMAIL{}
\newcommand\score{%
\vspace{-0.6in}
\begin{flushright}
Score: \answerbox{\POINTS}
\end{flushright}
\vspace{-0.4in}
\hspace{0.7in}\AUTHOR
\vspace{0.2in}
}

\newcommand\blankline{\mbox{}\\ }

\newcommand\ANSWER{\textsc{Answer:}\vspace{-2mm}}

\newcommand\LATEXHELPTHREEFIVEZERO{
In the \texttt{answerlong}, you will need to enter \LaTeX\ code for
mathematical notation.
Some incomplete or wrong answers are included in the \texttt{answerlong}
so that all you need to do is to make minor modifications.
Note that \texttt{answercode} is for writing code/pseudocode and
does not require mathematical notation.

Here are some pointers on writing math \LaTeX\ code:
\begin{enumerate}[nosep]

\item For \lq\lq inline math mode", use \texttt{\$...\$}.
Example: \texttt{\$x\ =\ 42\ +\ y\$} gives you $x = 42 + y$.
(Mathematical expressions have their own spacing, special symbols,
and are in italics.)

\item For \lq\lq display math mode", use \texttt{\textbackslash[...\textbackslash]}.
Example: \texttt{\textbackslash[\ x\ =\ 42 \textbackslash]} gives you \[ x = 42 \]
(Display math mode is used for emphasis.)

\item Here's how you do fractions:
\texttt{\$\textbackslash frac\{1\}\{2\}\$} gives you $\frac{1}{2}$.

\item Here's how you do subscript:
\texttt{\$t\_\{123\}\$} gives you $t_{123}$. 

\item Here's how you do superscript:
\texttt{\$n\^{}\{123\}\$} gives you $n^{123}$.
\end{enumerate}

The above information should be enough for this quiz.
For more information on \LaTeX\, you can go to
\href{http://bit.ly/yliow0/}{my website},
click on Yes you are one of my students, scroll down to the Tutorials
section and click on latex.pdf.
}

\newcommand\LATEXHELPTHREEFIVEZEROB{
In the \texttt{answerlong}, you will need to enter \LaTeX\ code for
mathematical notation.
Some incomplete or wrong answers are included in the \texttt{answerlong}
so that all you need to do is to make minor modifications.
Note that \texttt{answercode} is for writing code/pseudocode and
does not require mathematical notation.

Here are some pointers on writing math \LaTeX\ code:
\begin{enumerate}[nosep]

\item For \lq\lq inline math mode", use \texttt{\$...\$}.
Example: \texttt{\$x\ =\ 42\ +\ y\$} gives you $x = 42 + y$.
(Mathematical expressions have their own spacing, special symbols,
and are in italics.)

\item For \lq\lq display math mode", use \texttt{\textbackslash[...\textbackslash]}.
Example: \texttt{\textbackslash[\ x\ =\ 42 \textbackslash]} gives you \[ x = 42 \]
(Display math mode is used for emphasis.)

\item Here's how you do fractions:
\texttt{\$\textbackslash frac\{1\}\{2\}\$} gives you $\frac{1}{2}$.

\item Here's how you do subscript:
\texttt{\$t\_\{123\}\$} gives you $t_{123}$. 

\item Here's how you do superscript:
  \texttt{\$n\^{}\{123\}\$} gives you $n^{123}$.
  
\item Here's how you do log:
  \texttt{\$\textbackslash lg n\$} gives you $\lg n$.
\end{enumerate}

The above information should be enough for this quiz.
For more information on \LaTeX\, you can go to
\href{http://bit.ly/yliow0/}{my website},
click on Yes you are one of my students, scroll down to the Tutorials
section and click on latex.pdf.
}

