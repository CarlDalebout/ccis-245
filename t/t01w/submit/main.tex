\input{thispreamble.tex}

\renewcommand\AUTHOR{cadalebout1@cougars.ccis.edu} % CHANGE TO YOURS

\begin{document}
\topmattertwo

\input{preamble-instructions.tex}

\begin{center}
  \textsc{Honor Statement}
\end{center}
I, \answerbox{Carl A. Dalebout},
attest to the fact that the submitted work is my own and
is not the result of plagiarism.
Furthermore, I have not aided another student in the act of
plagiarism.

% ------------------------------------------------------------------------------
\begin{python}
from scoretable import *
\end{python}
        
%-------------------------------------------------------------------------------
\newpage
\nextq
Complete the following program. You can only add to the given code.
You cannot delete or change what is given below.

\textsc{Answer:}
\begin{answercode}
#include <iostream>

void swap(int * p, int * q)
{
        int  temp = &p;
        &p = &q;
        &q = temp;
}


int main()
{
    int x = 0;
    int y = 1;

    swap(&x , &y);
    
    std::cout << x << ' ' << y << '\n'; // expected output: 1 0
    
    return 0;
}
\end{answercode}

% ------------------------------------------------------------------------------
\newpage
\nextq
Complete the following program.
You can only add to the given code. 
You cannot delete or change what is given below.

The function \verb!reverse(x, x_len)! reverses the values in the integer
array \verb!x!
from index \verb!0! up to index \verb!x_len - 1!.
For instance suppose the user enters \verb!2 3 5 7 -9999!, then
the values \verb!2, 3, 5, 7! are placed in \verb!x!
and \verb!x_len! is set to 4.
After calling \verb!reverse(x, x_len)!,
the first four value in \verb!x! becomes \verb!7, 5, 3, 2!.

\textsc{Answer:}
\begin{answercode}
#include <iostream>

void swap(int x, int y)
{
        int temp = x;
        x = y;
        y = temp;
}

void reverse(int x[], int x_len)
{
        for(int i = 0; i <= x_len/2; ++i)
        {
                swap(x[i], x[x_len-i);
        }
}

int main()
{
    int x[1024];
    int x_len;

    x_len = 0;
    for (int i = 0; i < 1024; ++i)
    {
        int t;
        std::cin >> t;
        if (t == -9999)
        {
            break;
        }
        x[x_len] = t;
        ++x_len;
    }

    reverse(x, x_len);
    // x from index 0 to x_len - 1 is now reverse.
    
    return 0;
}
\end{answercode}

% ------------------------------------------------------------------------------
\newpage
\nextq
What is the output of this program?
\begin{Verbatim}[frame=single]
int x = 10; // Address 1000
int y = 20; // Address 1004
int * p = &x; // p = 1000 
int * q = &y; // q = 1004
*p = *p * *q; // 1000 = 10 * 20 or x = 10 * 20
*q = 42; // y = 42
std::cout << x << ' ' << y << '\n';
std::cout << *p << ' ' << *q << '\n';
\end{Verbatim}

\textsc{Answer:}
\begin{answercode}
200 42
200 42
\end{answercode}

%-------------------------------------------------------------------------------
\newpage
\nextq
Rewrite the following code fragment by doing the following: Do a simple addition
program using the following skeleton.
You MUST follow the given instructions.

\textsc{Answer:}
\begin{answercode}
int x = 0, y = 0;
int * p = &x;
int * q = &y;
// Do NOT declare any variables below.

// Statement to prompt the user for an integer and put it into x.
// Do NOT use x in your statement.
std::cout << " please give me a number for X " << '\n';
std::cin >> *p;


// Statement to prompt the user for an integer and put it into y. 
// Do NOT use y in your statement.
std::cout << " please give me a number for Y " << '\n';
std::cin >> *q;


std::cout << x + y << '\n';
\end{answercode}

%-------------------------------------------------------------------------------
\newpage
\nextq
What is the output of this code fragment?
\begin{Verbatim}[frame=single]
int i = 1; //Address 1000
int j = 2; //Address 1004
int k = 3; //Address 1008

int * p = &i; // p = 1000
int * q = &j; // q = 1004
int * r = &k; // r = 1008

p = q; p = 1004
q = r; q = 1008

std::cout << *p << ' ' << *q << ' ' << *r << '\n'; 
\end{Verbatim}

\textsc{Answer:}
\begin{answercode}
2 3 3
\end{answercode}

%-------------------------------------------------------------------------------
\newpage
\nextq
You were brainstorming with your team in one of the company's meeting rooms.
Your boss popped in to say hi on his way to get coffee and
he noticed the following diagram on the whiteboard.
Someone was tracing a piece of code on the whiteboard:
\begin{center}
  \includegraphics[width=3in]{pic1.PNG}
\end{center}
On his way back, your boss glanced at the whiteboard and saw this:
\begin{center}
  \includegraphics[width=3in]{pic2.PNG}
\end{center}
You noticed he was shaking his head as he walked away. Why?

\textsc{Answer:}
\begin{answercode}
because d is a was a int pointer and is not pointing at a double value.
\end{answercode}

%-------------------------------------------------------------------------------
\newpage
\nextq
The following program does compile and does run. But it has a memory leak.
Fix it so that there is no memory leak.

\textsc{Answer:}
\begin{answercode}
#include <iostream>

int sum(int n)
{
    int s = 0;
    int * i = new int;
    for (*i = 0; *i <= n; ++(*i))
    {
        s += *i;
    }
    delete i;
    return s;
}

int main()
{
    std::cout << sum(10) << '\n';
    return 0;
}
\end{answercode}

%-------------------------------------------------------------------------------
\newpage
\nextq
The following program prompts the user for two integer values
and then prints the sum.
Do NOT use integer or double variables -- you can only use pointers.
In fact I have already declared all the variables you need,
i.e., two pointer variables.
You must allocate and deallocate memory correctly.

\textsc{Answer:}
\begin{answercode}
#include <iostream>

int main()
{
    int * p;
    int * q;

    // allocate memory for p
    p = new int;

    // allocate memory for q
    q = new int;

    // prompt for integer value and store at integer that p points to
    std::cout << "please input an integer for p\n";
    std::cin >> *p;

    // prompt for integer value and store at integer that q points to
    std::cout << "please input an integer for p\n";
    std::cin >> *p;

    // print the sum of integers that p and q point to
    std::cout << "the summ of p and q is " << (*p + *q) << '\n';

    // deallocate memory used by q
    delete q;

    // deallcate memory used by p
    delete p;

    return 0;
}
\end{answercode}

%-------------------------------------------------------------------------------
\newpage
\nextq
Complete this code segment.

\textsc{Answer:}
\begin{answercode}
int x[] = {1, 5, 3, 7, 9, 4, 2, 6, 8, 0};
int max;

int * start = &x[0]; // 
int * end = &x[10];
int * p;
int * pmax = &max;

// Complete the following to compute the maximum value in the array and store
// it in variable max.
// Your code must work for different array values in x.
// You also cannot use the name x or max.
// You must use a loop (of course).
// You can only use integer pointer variables start, end, p, pmax;

*pmax = *start; // max = 1

for(p = start + 1; p < end ++p) // p = &x[1] p < &x[10]
{
        if(*p > *pmax) *pmax = *p;
}


// At this point the maximum value of x is stored in variable max.
std::cout << max << '\n';
\end{answercode}

%-------------------------------------------------------------------------------
\newpage
\nextq
The following have a function that attempts to perform memory allocation
and memory deallocation, but
they are done incorrectly:
\vspace{-3mm}
\begin{Verbatim}[frame=single,fontsize=\small]
void mynew(int * p)
{
    p = new int;
}

void mydelete(int * p)
{
    delete p;
    p = NULL;
}

int main()
{
    int * p;
    mynew(p);
    *p = 42;
    mydelete(p);
    return 0;
}
\end{Verbatim}
\vspace{-5mm}Fix the above problem below.

\textsc{Answer:}\vspace{-3mm}
\begin{answercode}
void mynew(int ** p)
{
        p = new int;
}

void mydelete(int ** p)
{
        delete p;
        p = NULL;
}

int main()
{
    int * p;
    mynew(&p);
    *p = 42;
    mydelete(&p);
    return 0;
}
\end{answercode}

%-------------------------------------------------------------------------------
\newpage
\nextq
Complete the following by writing a struct.
\\
\textsc{Answer:}\vspace{-2mm}
\begin{answercode}
#include <iostream>

// define the struct here
struct Student
{
        int student_id;
        int dob_year;
        int dob_month;
        int dob_day;
        double hight;
        double weight;
}



void input(Student & x)
{
    std::cout << "please provide your student id \n";   
    std::cin >> x.student_id; // prompt for an integer value for x's 
                              // student id
    std::cout << "please provide your birth year \n";                          
    std::cin >> x.dob_year;   // prompt for an integer value for x's year 
                              // of date of birth
    std::cout << "please provide your birth month \n";
    std::cin >> x.dob_month;  // prompt for an integer value for x's month 
                              // of date of birth
    std::cout << "please provide your day of birth \n";
    std::cin >> x.dob_day;    // prompt for an integer value for x's day 
                              // of date of birth
    std::cout << "please provide your hight and weight \n";
    std::cin >> x.height;     // prompt for x's height (which is a double)
    std::cin >> x.weight;     // prompt for x's weight (which is a double)
}

int main()
{
    Student john;
    input(john);
    return 0;
}
\end{answercode}

%-------------------------------------------------------------------------------
\newpage
\nextq
You are writing a tic-tac-toe game. The following code is in your
\verb!main()!:
\vspace{-3mm}
\begin{Verbatim}[frame=single,fontsize=\small]
#include <iostream>
#include "TTT.h"

int main()
{
    TTT board;

    while (1)
    {
        print(board);
        int row, col;
        get_input(board, row, col);
        make_move(board, row, col);
        if (game_ended(board)) 
        {
            break;
        }
    }
    print_result(board);

    return 0;
}
\end{Verbatim}
\vspace{-4mm}
Complete the header file (with the struct definition and the function
prototypes -- no function body definitions).
The struct and function prototypes must be minimal
(i.e., no useless member variables, no unnecessary parameters,
reference parameters must be constant whenever possible).

\textsc{Answer:}\vspace{-2mm}
\begin{answercode}
#ifndef TTT_H
#define TTT_H

struct TTT
{
int row[3];
int col[3];
}

void print(TTT &);

void get_input(TTT &, int, int);

void make_move(TTT, int, int);

bool (TTT &);

void print_result(TTT &);

#endif
\end{answercode}

%-------------------------------------------------------------------------------
\newpage
\nextq
What is the output? Or is there an error?
\begin{Verbatim}[frame=single,fontsize=\small]
#include <iostream>

int h(int * p)
{
    return *p;
}

int * g(int * p)
{
    return p;
}

int * f(int * p)
{
    return (p != NULL ? g(p) : NULL);
}

int main()
{
    int i = 5;
    std::cout << *f(&i) + h(&i) << std::endl;
    return 0;
}
\end{Verbatim}

\textsc{Answer:}\vspace{-2mm}
\begin{answercode}
10
\end{answercode}

%-------------------------------------------------------------------------------
\newpage
\nextq
Complete the following program. Make sure there is no memory leak.

\textsc{Answer:}\vspace{-3mm}
\begin{answercode}
#include <iostream>
#include <cmath>
int f(int n)
{
    int * p;

    // Allocate an integer array of size n to p. (Of course the array
    // is in the heap.)
    p = new int[n];

    // Fill the array that p points to with values 1, 2, 3, ..., n.
    for(int i = 0; i < n; ++i)
    {
        p[i] = i+1;
    }

    // Go over the values in the array that p points to and 
    // (1) if a value is odd, replace that value by the square root of the
    //     value, or
    // (2) if a value x is even, replace that value x by x + 1. 
    // This is one pass.
    // Repeat this until every value in the array is <= 42.
    // Return the number of passes you have to run over the array

    int ret; // number of passes
    bool flag = false;
    while(1)
    {
        for(int i = 0; i < n; ++i)
        {
                flag = true
                if(p[i] % 2 = 1)
                {
                        p[i] = sqrt(p[i]);
                        if(p[i] > 42) flag = false;
                }
                else
                {
                        p[i] += 1;
                        if(p[i] > 42) flag = false;
                }
        }
        ++ret;
        if(flag) return ret;        
    }
}

int main()
{
    int n;
    std::cin >> n;
    std::cout << f(n) << '\n';
    return 0;
}
\end{answercode}

%-------------------------------------------------------------------------------
\newpage
\nextq
Complete the following function that performs the binary search.
You need NOT use recursion.
    
\textsc{Answer:}\vspace{-2mm}
\begin{answercode}
// Performs binary search on *start, *(start+1), ..., *(end - 1) for the
// value of target and return the pointer where target is found.
// If target is not found, NULL is returned.
int * binarysearch(int * start, int * end, int target)
{
        
        if (target > *((end-1)/2))
        {
        binarysearch((end/2), end, target)
        }
        if(target < (((end-1)/2)))
        {
        binarysearch(start,(end/2));
        }
        if(start == end)return *start;
        if(start > end) return NULL;

}
\end{answercode}


%------------------------------------------------------------------------------
\newpage
\input{instructions}
\end{document}