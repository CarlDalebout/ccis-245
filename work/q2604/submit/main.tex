\input{thispreamble.tex}

\renewcommand\AUTHOR{cadalebout1@cougars.ccis.edu} % CHANGE TO YOURS

\begin{document}
\topmattertwo


\nextq
\verb!p! is a pointer to a \verb!double!.
Write one statement allocating memory (one \verb!double!) in the heap
for \verb!p! to point to.
\\ \textsc{Answer:}\vspace{-2mm}
\begin{answercode}
double * p = new double;
\end{answercode}

%------------------------------------------------------------------------------

\nextq
\verb!q! is a pointer to a \verb!char!.
Memory (1 character) has already been allocated \verb!q!.
Write one statement to deallocate the memory used by \verb!q!.
\\ \textsc{Answer:}\vspace{-2mm}
\begin{answercode}
delete q;
\end{answercode}

%------------------------------------------------------------------------------
\nextq
Write a code fragment that does the following:
Declare a pointer \verb!r! as a pointer to an \verb!int!.
Allocate memory (one \verb!int!) for \verb!r! using the heap.
Get an integer from the user and store that value in the integer
that \verb!r! points to.
Increment the value that \verb!r! points to by 5.
Print the value that \verb!r! points to.
Finally deallocate the memory used by \verb!r!.
There should be 6 statements and
there must be only one variable declared in your code fragment.
\\ \textsc{Answer:}\vspace{-2mm}
\begin{answercode}
int *r = new int;
std::cin >> *r;
*r += 5;
std::cout << *r << std::endl;
delete r;
\end{answercode}


\newpage
\input{instructions.tex}
\end{document}
