\newcommand\COURSE{ciss245}
\newcommand\ASSESSMENT{q2604}
\newcommand\ASSESSMENTTYPE{Quiz}
\newcommand\POINTS{\textwhite{xxx/xxx}}

\makeatletter
\DeclareOldFontCommand{\rm}{\normalfont\rmfamily}{\mathrm}
\DeclareOldFontCommand{\sf}{\normalfont\sffamily}{\mathsf}
\DeclareOldFontCommand{\tt}{\normalfont\ttfamily}{\mathtt}
\DeclareOldFontCommand{\bf}{\normalfont\bfseries}{\mathbf}
\DeclareOldFontCommand{\it}{\normalfont\itshape}{\mathit}
\DeclareOldFontCommand{\sl}{\normalfont\slshape}{\@nomath\sl}
\DeclareOldFontCommand{\sc}{\normalfont\scshape}{\@nomath\sc}
\makeatother

\input{myquizpreamble}
\input{yliow}
\input{\COURSE}
\textwidth=6in

\renewcommand\TITLE{\ASSESSMENTTYPE \ \ASSESSMENT}

\newcommand\topmattertwo{
\topmatter
\score \\ \\
Open \texttt{main.tex} and enter answers (look for
\texttt{answercode}, \texttt{answerbox}, \texttt{answerlong}).
Turn the page for detailed instructions.
To rebuild and view pdf, in bash shell execute \texttt{make}.
To build a gzip-tar file, in bash shell execute \texttt{make s} and
you'll get \texttt{submit.tar.gz}.
}

\newcommand\tf{T or F or M}
\newcommand\answerbox[1]{\textbox{\phantom{|}\hspace{-4mm}#1}}
\newcommand\codebox[1]{\begin{console}#1\end{console}}

\usepackage{pifont}
\newcommand{\cmark}{\textred{\ding{51}}}
\newcommand{\xmark}{\textred{\ding{55}}}

\newcounter{qc}
\newcommand\nextq{
%\newpage
\addtocounter{qc}{1}
Q{\theqc}.
}

\DefineVerbatimEnvironment%
 {answercode}{Verbatim}
 {frame=single,fontsize=\footnotesize}

\newenvironment{largebox}[1]{%
 \boxparone{#1}
}
{}

\usepackage{environ}
\let\oldquote=\quote
\let\endoldquote=\endquote
\let\quote\relax
\let\endquote\relax

% ADDED 2021/09/09
\renewcommand\boxpar[1]{
 \[
  \framebox[\textwidth][c] {
   \parbox[]{\dimexpr\textwidth - 0.25cm} {#1}
  }
 \]
}

\NewEnviron{answerlong}%
  {\vspace{-1mm} \global\let\tmp\BODY\aftergroup\doboxpar}

\newcommand\doboxpar{%
  \let\quote=\oldquote
  \let\endquote=\endoldquote
  \boxpar{\tmp}
}

\newenvironment{mcq}[7]%
{% begin code
#1 \dotfill{#2}
 \begin{tightlist}
 \item[(A)] #3
 \item[(B)] #4
 \item[(C)] #5
 \item[(D)] #6
 \item[(E)] #7 
 \end{tightlist}
}%
{% end code
} 

\renewcommand\EMAIL{}
\newcommand\score{%
\vspace{-0.6in}
\begin{flushright}
Score: \answerbox{\POINTS}
\end{flushright}
\vspace{-0.4in}
\hspace{0.7in}\AUTHOR
\vspace{0.2in}
}

\newcommand\blankline{\mbox{}\\ }

\newcommand\ANSWER{\textsc{Answer:}\vspace{-2mm}}

\newcommand\LATEXHELPTHREEFIVEZERO{
In the \texttt{answerlong}, you will need to enter \LaTeX\ code for
mathematical notation.
Some incomplete or wrong answers are included in the \texttt{answerlong}
so that all you need to do is to make minor modifications.
Note that \texttt{answercode} is for writing code/pseudocode and
does not require mathematical notation.

Here are some pointers on writing math \LaTeX\ code:
\begin{enumerate}[nosep]

\item For \lq\lq inline math mode", use \texttt{\$...\$}.
Example: \texttt{\$x\ =\ 42\ +\ y\$} gives you $x = 42 + y$.
(Mathematical expressions have their own spacing, special symbols,
and are in italics.)

\item For \lq\lq display math mode", use \texttt{\textbackslash[...\textbackslash]}.
Example: \texttt{\textbackslash[\ x\ =\ 42 \textbackslash]} gives you \[ x = 42 \]
(Display math mode is used for emphasis.)

\item Here's how you do fractions:
\texttt{\$\textbackslash frac\{1\}\{2\}\$} gives you $\frac{1}{2}$.

\item Here's how you do subscript:
\texttt{\$t\_\{123\}\$} gives you $t_{123}$. 

\item Here's how you do superscript:
\texttt{\$n\^{}\{123\}\$} gives you $n^{123}$.
\end{enumerate}

The above information should be enough for this quiz.
For more information on \LaTeX\, you can go to
\href{http://bit.ly/yliow0/}{my website},
click on Yes you are one of my students, scroll down to the Tutorials
section and click on latex.pdf.
}

\newcommand\LATEXHELPTHREEFIVEZEROB{
In the \texttt{answerlong}, you will need to enter \LaTeX\ code for
mathematical notation.
Some incomplete or wrong answers are included in the \texttt{answerlong}
so that all you need to do is to make minor modifications.
Note that \texttt{answercode} is for writing code/pseudocode and
does not require mathematical notation.

Here are some pointers on writing math \LaTeX\ code:
\begin{enumerate}[nosep]

\item For \lq\lq inline math mode", use \texttt{\$...\$}.
Example: \texttt{\$x\ =\ 42\ +\ y\$} gives you $x = 42 + y$.
(Mathematical expressions have their own spacing, special symbols,
and are in italics.)

\item For \lq\lq display math mode", use \texttt{\textbackslash[...\textbackslash]}.
Example: \texttt{\textbackslash[\ x\ =\ 42 \textbackslash]} gives you \[ x = 42 \]
(Display math mode is used for emphasis.)

\item Here's how you do fractions:
\texttt{\$\textbackslash frac\{1\}\{2\}\$} gives you $\frac{1}{2}$.

\item Here's how you do subscript:
\texttt{\$t\_\{123\}\$} gives you $t_{123}$. 

\item Here's how you do superscript:
  \texttt{\$n\^{}\{123\}\$} gives you $n^{123}$.
  
\item Here's how you do log:
  \texttt{\$\textbackslash lg n\$} gives you $\lg n$.
\end{enumerate}

The above information should be enough for this quiz.
For more information on \LaTeX\, you can go to
\href{http://bit.ly/yliow0/}{my website},
click on Yes you are one of my students, scroll down to the Tutorials
section and click on latex.pdf.
}



\renewcommand\AUTHOR{cadalebout1@cougars.ccis.edu} % CHANGE TO YOURS

\begin{document}
\topmattertwo

When submitting using alex, enter \verb!f01! for work.

\textsc{Instructions}
\begin{enumerate}
\li This is a closed-book, no-discussion, no-calculator, no-browsing-on-the-web
    no-compiler/no-MIPS-simulator test.
\li Cheating is a serious academic offense. If caught you will 
    receive an immediate score of -100\%.
\li If a question asks for a program output and the program or
    code fragment contains
    an error, write \verb!ERROR! as output.
    When writing output, whitespace is significant.
\li If a question asks for the computation of a value and the program or
    code fragment contains
    an error, write \verb!ERROR! as value.
\li When you're asked to write a C++ statement, don't forget that it must
    end with a semicolon.
\li Bubblesort refers to the bubblesort algorithm in our notes
    where values are sorted in ascending order.
\end{enumerate}

\vspace{1cm}


\begin{center}
  \textsc{Honor Statement}
\end{center}
I, \answerbox{[Carl Alexander Dalebout]},
attest to the fact that the submitted work is my own and
is not the result of plagiarism.
Furthermore, I have not aided another student in the act of
plagiarism.

% ------------------------------------------------------------------------------
\begin{python}
from scoretable import *
\end{python}
        
%-------------------------------------------------------------------------------
\newpage
\nextq
Consider the following program.
What is the output of this program if the user enters
\verb!0! for \verb!i! and \verb!1! for \verb!j!. 
\begin{console}
int i, j;
std::cin >> i >> j;
int * p = &i;
int * q = &j;
*p += *q;
*q = *p + 1;
std::cout << i << ' ' << j << '\n';
\end{console}
Complete the following program. You can only add to the given code.
You cannot delete or change what is given below.

\textsc{Answer:}
\begin{answercode}
1 2
\end{answercode}


%------------------------------------------------------------------------------
\newpage
\nextq
What is the output of this code fragment
if the user enters \verb!0! for \verb!i!, \verb!1! for \verb!j!,
and \verb!2! for \verb!k!?
\begin{console}
int i;
int j;
int k;
std::cin >> i >> j >> k;
int * p = &i;
int * q = &j;
int * r = &k;
q = p;
p = r;
std::cout << *p << ' ' << *q << ' ' << *r << '\n'; 
\end{console}

\textsc{Answer:}
\begin{answercode}
2 0 2
\end{answercode}


%------------------------------------------------------------------------------
\newpage
\nextq
The following code fragment has a memory leak.
Correct the code to ensure there's no memory leak.
(Make corrections to the code below directly.)

\textsc{Answer:}
\begin{answercode}
int f(int n)
{
    int * a = new int[n];
    int * i = new int;

    a[0] = rand();

    for (*i = 1; *i < n; ++(*i))
    {
        a[*i] += rand();
        a[*i] -= a[*i - 1];
    }
    int ret = a[n - 1];

    delete a;
    a = NULL;
    delete[] i;
    i = NULL;

    return ret;
}
\end{answercode}


%-------------------------------------------------------------------------------
\newpage
\nextq
The following program computes and prints the minimum of an array of 10 integer
values entered by the user.
Follow the instructions (in comments) in order to complete the program.
Do not remove the comments.
I have already declared all the variables you need.
Therefore do not declare any other variables.
You must allocate and deallocate memory correctly.

\textsc{Answer:}
\begin{answercode}
#include <iostream>

int main()
{
    int * p;
    int * q;
    int * min; 

    // Allocate memory for p so that p points to an array of 10 values
    // in the memory heap.
    p = new int[10];

    // Prompt user for 10 integers and put them in the array that p
    // points to.
    std::cout << "please give me 10 integers\n";

    for (q = p; q < p + 10; ++q)
    {
        std::cin >> *q;
        if(q == p)
             min = q;
        else if( *q < *min )
             min = q;
    }

    // Compute the minimum of values in the array of 10 values that p
    // points to and store in integer that min points to.
    

    std::cout << *min << std::endl;

    // Deallcate memory used by p
    delete[] p;
    p = NULL;
    
    return 0;
}
\end{answercode}


%-------------------------------------------------------------------------------
\newpage
\nextq
Write a function that checks if a string is a palindrome.
A palindrome is a string that is it’s own \lq\lq reverse//".
In other words, you read the string left-to-right and right-to-left,
you get the same string. For instance
\[
	\texttt{"madam"}
\]
Another example is 
\[
	\texttt{"tacocat"}
\]

Don’t forget that a C-string is null-terminated, i.e.,
there’s an end-of-data character of \verb!'\0'! in the string that indicates
where the string data ends.
You \textit{cannot} use any C-string functions or the C++ string class.

\textsc{Answer:}
\begin{answercode}
bool is_palindrome(char s[])
{
    int i = 0;
    while(s[i] != '\0')
        ++i;
    for(int j = 0; j < i-1; ++j)
    {
        if(j != i && s[j] != s[i-1])
            return false;
        --i;
    }
    return true;
}
\end{answercode}


%-------------------------------------------------------------------------------
\newpage
\nextq
Complete the following \verb!Rectangle! class given the following requirements:
\begin{enumerate}[nosep]

\li There are two private member variables \verb!width_! and \verb!height_!,
    both of \verb!double! type.

\li You can construct a \verb!Rectangle! object of a given width (say 10) and
    given height (say 5) as follows:
    \begin{Verbatim}[frame=single,fontsize=\small]
Rectangle r(10, 5);
    \end{Verbatim}
You must use an initializer list whenever possible.

\li You can also construct a \verb!Rectangle! object (say \verb!r1!) with
    another \verb!Rectangle! object (say \verb!r2!) so that
    \verb!r1! and \verb!r2! have the same width and height:
    \begin{Verbatim}[frame=single,fontsize=\small]
Rectangle r1(r2);
    \end{Verbatim}
    You must use an initializer list whenever possible.

\li You can get the value of the \verb!width_! of a Rectangle object
    (say \verb!r!) as follows:
    \begin{Verbatim}[frame=single,fontsize=\small]
std::cout << r.get_width() << std::endl;
    \end{Verbatim}

\li You can also get the value of the \verb!width_! of a
    \verb!Rectangle! object (say \verb!r!) as follows:
    \begin{Verbatim}[frame=single,fontsize=\small]
std::cout << r.width() << std::endl;
    \end{Verbatim}

\li You can set the value of the \verb!width_! of a \verb!Rectangle! object
    (say \verb!r!) as follows:
    \begin{Verbatim}[frame=single,fontsize=\small]
r.set_width(12.34);
    \end{Verbatim}

\li You can also set the value of the \verb!width_! of a \verb!Rectangle!
    object (say \verb!r!) as follows:
    \begin{Verbatim}[frame=single,fontsize=\small]
r.width() = 12.34;
    \end{Verbatim}
    
\li You can also get the area of a \verb!Rectangle! object (say \verb!r!) as
    follows:
    \begin{Verbatim}[frame=single,fontsize=\small]
std::cout << r.area() << std::endl;
    \end{Verbatim}
    
\end{enumerate}
All methods are implemented in the header file below.

(Turn page)

\newpage
\textsc{Answer:}
\begin{answercode}
// file: Rectangle.h

#ifndef RECTANGLE_H
#define RECTANGLE_H

class Rectangle
{
public:
    Rectangle(double width, double height)
        :width_(width), height_(height)
    {}
    Rectangle(Rechtangle & rec)
        :width_(rec.width_), hight_(rec.height_)
    {}

    double  get_width()   const { return width_;  }
    double  width()       const { return width_;  }
    double &width()             { return width_;  }
    void    set_width(double & i) { width_ = i; }

    double  get_height_() const { return height_; }
    double  height()      const { return height_; }
    double &height()            { return height_; }
    void    set_height(double & i) { height_ = i; }

    double  area()        const { return height_*width_; }
    
private:
    double width_, height_;
};

#endif
\end{answercode}


%-------------------------------------------------------------------------------
\newpage
\nextq
It has been decreed that every living thing in the Milky Way galaxy must have
an integer \verb!id! (an integer).
A human being of course is a living thing,
but a human being has a gender (a character) and number of heads, arms, and
legs.
Create two classes: \verb!LivingThing! and \verb!Earthling! where
\verb!Earthling! is a subclass of \verb!LivingThing!.
You only need to have enough code for the following code to execute:
\begin{console}[fontsize=\small]
// john has id 3423452, is male, has 2 heads, 3 arms, and 4 legs
Earthling john(3423452, 'm', 2, 3, 4);
\end{console}
All member variables must be private.
Furthermore, based on the information above, various private member variables
must be placed in the right class to minimize code duplication.
(for instance if one were to add a class for martians).

Implement the \verb!LivingThing! class below.
(See next question for the \verb!Earthling! class.)

\textsc{Answer:}
\begin{answercode}
// file: LivingThing.h

#ifndef LIVINGTHING_H
#define LIVINGTHING_H

class LivingThing
{
public:
    LivingThing(int id, char sex, int heads, int arms, int legs)
        :id_(id), sex_(sex), heads_(heads), arms_(arms), legs_(legs)
    {}
private:
    int id_;
    char sex_;
    int heads_, arms_, legs_;
};

#endif
\end{answercode}


%-------------------------------------------------------------------------------
\newpage
\nextq
This is a continuation of the previous question.
Implement the \verb!Earthling! class below.

\textsc{Answer:}
\begin{answercode}

// file: Earthling.h
#ifndef EARTHLING_H
#define EARTHLING_H

#include "LivingThings.h"

class Earthling: public LivingThings
{
public:
    Earthling(int id, char sex, int heads, int arms, int legs)
        :LivingThings(id, sex, heads, arms, legs)
    {}
private:
    
};

#endif
\end{answercode}


%-------------------------------------------------------------------------------
\newpage
\nextq
The following will print \verb!"A::f()\n"! twice.
Modify it so that one of the outputs is
\verb!"A::f()\n"! and the other is \verb!"B::f()\n"!. 

\textsc{Answer:}
\begin{answercode}
#include <iostream>

class A
{
public:
    virtual void f() const
    {
        std::cout << "A::f()\n";
    }
};

class B: public A
{
public:
    void f() const
    {
        std::cout << "B::f()\n";
    }
};

int main()
{
    B b1;
    A a1 = b1; 
    a1.f();

    B b2;
    A * a2 = &b2;
    a2->f();

    return 0;
}
\end{answercode}

%-------------------------------------------------------------------------------
\newpage
\nextq
The following class allows us to create objects that models rational numbers:
\begin{console}
class Rational
{
public:
    Rational(int n, int d)
        : n_(n), d_(d)
    {}
    // other methods not shown
private:
    int n_;
    int d_;
};
\end{console}
Complete two methods that allows you to perform additions
(\verb!+! and \verb!+=!).
As an example, the following usage of the Rational class should work correctly:
\begin{console}
Rational r0(1, 2), r1(3, 4); // r0 models 1/2 and r1 models 3/4
Rational r2 = r0 + r1;       // r2 models 1/2 + 3/4 , 
                             // i.e., (1 * 3 + 2 * 4) / (2 * 4)
Rational r3 = r2;
r3 += r2;                    // r3 models r3 + r2
\end{console}
Your code \textit{must} be minimal.
(HINT: \verb!+! should use \verb!+=!.)

(Turn page)

\newpage
\textsc{Answer:}
\begin{answercode}
// file: Rational.h

#ifndef RATIONAL_H
#define RATIONAL_H

class Rational
{
public:
    Rational(int n, int d)
        : n_(n), d_(d)
    {}
    Rational(Rational & rat)
        :n_(rat.n_), d_(rat.d_)
    {}

    Rational operator+=(Rational & rat)
    {
        n_ = n_*rat.d_ + rat.n_ * d_;
        d_ *= rat.d_;

        return *this;        
    }
    Rational operator+ (Rational & rat)
    {
        Rational temp = *this;
        return temp += rat; 
    }
    
private:
    int n_;
    int d_;
};

#endif
\end{answercode}


%-------------------------------------------------------------------------------
\newpage
\nextq
The following class allows us to create objects that handle dynamic array of
integers, i.e., each object contains a pointer to an array in the heap:
\begin{console}[fontsize=\small]
class IntDynArr
{
public:
    IntDynArr(int size)
        : p_(new int[size]), size_(size), capacity_(size)
    {}
    // other methods not shown
private:
    int * p_;
    int size_;
    int capacity_;
};
\end{console}
Of course since objects of the class uses a resource that is not automatically
released, you have to write the
\begin{enumerate}[nosep]
\li Destructor
\li Copy constructor
\li Assignment operator
\end{enumerate}
in order to overwrite the default ones:
\begin{console}[fontsize=\small]
class IntDynArr
{
public:
    IntDynArr(int size)
        : p_(new int[size]), size_(size), capacity_(size)
    {}

    ~IntDynArr();
    IntDynArr(const IntDynArr &);
    IntDynArr & operator=(const IntDynArr &);

    // other methods not shown
private:
    int * p_;
    int size_;
    int capacity_;
};
\end{console}

Implement the destructor below.
(The copy constructor and assignment operator are in the next two
questions.)

\textsc{Answer:}
\begin{answercode}
// file: IntDynArray.cpp

#include "IntDynArray.h"

IntDynArray::~IntDynArr()
{
    delete[] p_;
    p_ = NULL;
}

// other methods not shown
\end{answercode}


%-------------------------------------------------------------------------------
\newpage
\nextq
This is a continuation of the previous question.
Implement the copy constructor below for \verb!IntDynArray!.
(Constructors not using initializer lists as much as possible are considered
incorrect.)

\textsc{Answer:}
\begin{answercode}
// file: IntDynArray.cpp

#include "IntDynArray.h"

IntDynArray::IntDynArr(const IntDynArray & x)
    :p_(new int[x.size_]), size_(x.size_), capacity_(x.capacity_)
{
    for(int i = 0; i < size_; ++i)
    {
        p_ = x.p_;
    }
}

// other methods not shown
\end{answercode}


%-------------------------------------------------------------------------------
\newpage
\nextq
This is a continuation of the previous question.
Implement the assignment operator below for \verb!IntDynArray!.

\textsc{Answer:}
\begin{answercode}
// file: IntDynArray.cpp

#include "IntDynArray.h"

IntDynArr & IntDynArray::operator=(cosnt IntDynArray & x)
{
    size_ = x.size_;
    capacity = x.capacity_;
    for(int i = 0; i < size_; ++i)
    {
        p_ = x.p_;
    }
}

// other methods not shown
\end{answercode}


%-------------------------------------------------------------------------------
\newpage
\nextq
Complete the following recursive function \verb!sumsquares()! so that
\verb!sumsquares(n)! computes $0^2 + 1^2 + 2^2 + \cdots + n^2$, i.e.,
    it computes the \lq\lq sum of squares//".
$n$ is an integer that is at least $0$.

\textsc{Answer:}
\begin{answercode}
int sumsquares(int n)
{
    if(n == 0)
        return 0;
    else
        return sumsquares(n-1) + (n * n);
}
\end{answercode}


%-------------------------------------------------------------------------------
\newpage
\nextq
Write a recursive function to print the integers from address
\verb!start! up to \textit{but not including} address \verb!end!.
Between every two integers, the function prints a space.
After all integers are print, the function prints a newline character.

\textsc{Answer:}
\begin{answercode}
void println(int * start, int * end)
{
    if(start == end - 1)
        std::cout << *start;
    else
    {
        println(start, end - 1);
        std::cout << ' ' << end - 1; 
    }
}
\end{answercode}

%-------------------------------------------------------------------------------
\newpage
\nextq
Convert the following \verb!vec2d!, which models a vector in 2D space with
two \verb!doubles! to a class template,
which models a vector in 2D space with two \verb!T! values.
You only need to modify what's in the given class.
Do not implement any extra methods/functions not shown below.

\textsc{Answer:}
\begin{answercode}
// file: vec2d.h

#ifndef VEC2D_H
#define VEC2D_H

template < typename T>
class vec2d
{
public:
    vec2d(T x, T y)
        : x_(x), y_(y)
    {}
    double operator[](int i) const
    {
        return (i == 0 ? x_ : y_);
    }
    double & operator[](int i)
    {
        return (i == 0 ? x_ : y_);
    }
private:
    T x_;
    T y_;
};

#endif
\end{answercode}


%-------------------------------------------------------------------------------
\newpage
\nextq
The class belows allows me to do this:
\begin{console}[fontsize=\small]
Square square;
std::cout << square.compute(5) << '\n'; // prints 25
\end{console}
Modify the only method in the class so that I get a functor, i.e.,
so that I can do this instead:
\begin{console}[fontsize=\small]
Square square;
std::cout << square(5) << '\n'; // prints 25
\end{console}

\textsc{Answer:}
\begin{answercode}
// file: square.h

#ifndef SQUARE_H
#define SQUARE_H

class Square
{
public:
    Square(int x)
    {
        return x * x;
    }
};

#endif
\end{answercode}

%------------------------------------------------------------------------------
\newpage
\textsc{Instructions}

In \verb!main.tex! change the email address in
\begin{console}
\renewcommand\AUTHOR{jdoe5@cougars.ccis.edu} 
\end{console}
to yours.
In the bash shell, execute \lq\lq \verb!make!" to recompile \verb!main.pdf!.
Execute \lq\lq \verb!make v!" to view \verb!main.pdf!.
Execute \lq\lq \verb!make s!" to create \verb!submit.tar.gz! for submission.

For each question, you'll see boxes for you to fill.
You write your answers in \verb!main.tex! file.
For small boxes, if you see
\begin{console}[frame=single=single,fontsize=\small]
1 + 1 = \answerbox{}.
\end{console}
you do this:
\begin{console}[frame=single=single,fontsize=\small]
1 + 1 = \answerbox{2}.
\end{console}
\verb!answerbox! will also appear in
\lq\lq true/false" and \lq\lq multiple-choice"
questions.

For longer answers that needs typewriter font, if you see
\begin{console}[frame=single=single, fontsize=\small]
Write a C++ statement that declares an integer variable name x.
\begin{answercode}
\end{answercode}
\end{console}
you do this:
\begin{console}[frame=single=single, fontsize=\small]
Write a C++ statement that declares an integer variable name x.
\begin{answercode}
int x;
\end{answercode}
\end{console}
\verb!answercode! will appear in questions asking for
code, algorithm, and program output.
In this case, indentation and spacing is significant.
For program output, I do look at spaces and newlines.

For long answers (not in typewriter font) if you see
\begin{console}[frame=single=single, fontsize=\small]
What is the color of the sky?
\begin{answerlong}
\end{answerlong}
\end{console}
you can write
\begin{console}[frame=single=single, fontsize=\small]
What is the color of the sky?
\begin{answerlong}
The color of the sky is blue.
\end{answerlong}
\end{console}
For students beyond 245: You can put \LaTeX\ commands in
\verb!answerbox! and 
\verb!answerlong!.

A question that begins with \lq\lq T or F or M"
requires you to identify whether it is true or
false, or meaningless.
\lq\lq Meaningless" means something's wrong with the statement and
it is not well-defined.
Something like \lq\lq $1 +_2$" or \lq\lq $\{2\}^{\{3\}}$" is not
well-defined.
Therefore a question such as
\lq\lq Is $42 = 1 +_2$ true or false?" or
\lq\lq Is $42 = \{2\}^{\{3\}}$ true or false?"
does not make sense.
\lq\lq Is $P(42) = \{42\}$ true or false?" is meaningless because $P(X)$
is only defined if $X$ is a set.
For \lq\lq Is 1 + 2 + 3 true or false?", \lq\lq 1 + 2 + 3" is well--defined but
as a
\lq\lq numerical expression", not as a \lq\lq proposition", i.e.,
it cannot be true or false.
Therefore \lq\lq Is 1 + 2 + 3 true or false?" is also not a well-defined
question.

When writing results of computations, make sure it's simplified.
For instance write $2$ instead of $1 + 1$.
When you write down sets,
if the answer is $\{1\}$, I do not
want to see $\{1, 1\}$.

When writing a counterexample, always write the simplest.

Here are some examples (see \verb!instructions.tex! for details):

\begin{enumerate}

 \item \tf: 1 + 1 = 2 \dotfill\answerbox{T}
 
 \item \tf: 1 + 1 = 3 \dotfill\answerbox{F}
 
 \item \tf: $1 +^2 =$ \dotfill\answerbox{M}
 
 \item $1 + 2 =$ \answerbox{3}
 
 \item Write a C++ statement to declare an integer variable named
 \verb!x!.
 \begin{answercode}
int x;
 \end{answercode}

 \item Solve $x^2 - 1 = 0$.
 \begin{answerlong}
 Since $x^2 - 1 = (x-1)(x+1)$, $x^2 - 1 = 0$ implies $(x-1)(x+1)=0$.
 Therefore $x - 1 = 0$ or $x = -1$.
 Hence $x = 1$ or $x = -1$.
 \end{answerlong}

 \item
 \begin{mcq}
 {Which is true?}{\answerbox{C}}
 {$1+1=0$}
 {$1+1=1$}
 {$1+1=2$}
 {$1+1=3$}
 {$1+1=4$}
 \end{mcq}


\end{enumerate}

\end{document}